\chapter{Introduction}

\newacronym{sm}{SM}{Standard Model}

The quest to understand the universe at its most fundamental level drives the core of particle physics research. Researchers in this field aim to explore the properties, behaviors, and interactions of elementary particles. Particle physics seeks to answer fundamental questions about the universe, such as the origin of mass, the nature of dark matter, and the unification of forces.

The \acrlong{sm} is a theoretical framework that describes the fundamental particles and their interactions, excluding gravity. It encompasses three of the four known fundamental forces: electromagnetism, the weak nuclear force, and the strong nuclear force and explains the Higgs mechanism, giving particles mass. The particles in the Standard Model include quarks, leptons, gauge bosons, and the Higgs boson. Quarks and leptons are the building blocks of matter, while gauge bosons mediate the interactions between these particles.

Particle colliders, key instruments in this research process, accelerate and collide particles to reveal new insights through the analysis of the byproducts of such collisions. These events or processes are characterized by their cross sections. Often denoted as $\sigma$ and measured in units of area, the cross section of an event quantifies the probability of that event occurring in particle collisions. By comparing observed cross sections with those predicted by the \acrshort{sm}, physicists can test the validity of the model and potentially identify discrepancies indicating new physics beyond the \acrlong{sm}.

The efficacy of these colliders relies heavily on a critical factor: the luminosity, \Ilum. This parameter, quantifying the rate at which particles pass through a given area, directly influences the number of collisions a collider can produce. The instantaneous luminosity, \ilum, is a proportionality constant that relates the rate of an event, $R$, to its cross section, $R = \ilum \sigma$.

The collosal importance of precise cross section measurements, from which the \acrshort{sm} can be probed, sparks the need for precise measurements of the luminosity and fuels the motivation behind this thesis.